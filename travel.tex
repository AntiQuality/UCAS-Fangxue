\chapter{旅游}

\section{前言}
境外旅游和国内旅游差异不大,但也有需要特别注意的事项。其一,永远将人身安全放在第一位;其二,护照是在境外能证明身份的最重要的文件,请在访学和旅游全程妥善保管。如有闲暇时间建议考取加州驾照,具有5年有效期、可在美国境内作为租车、乘机、购票等的官方身份证、可换取IDP国际驾照。(补充:在2025年5月实行real ID政策之后,只有real ID版本的驾照可用作乘机证明。若因各种原因无法在机场提供身份证明,TSA可以通过问讯+核对系统的方式放行,但耗时较长。)虽然大部分租车公司承认中国驾照,但在可能的经济纠纷或交通事故中或成为不利因素。加州驾考流程和经验可查小红书,笔试直接前往 \href{https://www.dmv.ca.gov/}{加州车管所官网} 报名线上网考,路考可关注小红书“湾区谭教练”。其三,访学办称因安全问题不鼓励租车自驾,所以请务必熟知并遵守当地交通法规与驾驶习惯。其四,因为美国大部分地区公共交通不发达且人烟稀少,出行前最好结合本手册和小红书敲定大致行程并做好应急预案。特别地,以下所有信息均为笔者个人亲身经历和体验,主观性和时效性强,仅供参考。

\section{航空及铁路旅行提示}
本章节为后续修订时添加,时效位于2024年下半年,部分有注明的位置除外。

\subsection{航空旅行}
美国航空业发达,乘坐航班进行短途旅行如同在国内乘坐高铁一样,是十分大众化的选择。在购买境外机票时,本人推荐从航司官方网站直接购入,当然选择航班时可以使用国内各家OTA,以及google flight, skyscanner等进行筛选和比价。需要注意的一点是,同程和去哪儿在机票领域多次暴雷,出现过盗刷信用卡为用户出票的行为,谨慎选择。

购票时还需注意的一点是,务必查看好允许携带的行李额度。随身行李(personal items)、登机行李(carry-on items)和托运行李(checked items)是三个不同的概念,请根据自己需要携带的行李量进行选择。如果到机场才发现行李超过机票允许的额度,那恭喜你又为航司的营收贡献了神秘的东方力量。

票价方面,和国内一样,基本上越早买越便宜,尽量不要临期买票,尤其是假期的时候,真的很贵……

如果对收集登机牌没有什么执念,美国的机场绝大多数都支持直接使用电子登机牌完成安检到登机的全流程,此时你需要查看航司APP或者购票时预留的邮箱,提前进行线上值机获取电子登机牌。(如果你用了国内的OTA购票,大概率你收不到电子登机牌的邮件,因为他们会填写其他的邮箱地址——此时还是需要去值机柜台)

关于选座,如果对某种座位有执念,可以参考seatguru上的座位评分,根据航司与机型划分的检索方式十分便捷。但需要注意的是,部分航司选座需要付款。

以下为本人体验过的部分航司,优点与雷点因人而异,仅供参考:
\begin{itemize}
    \item F8 (Flair):中规中矩的加拿大廉价航空,经常爆SFO-YVR的低价,但这个价格确实是没有托运和登机行李,只能带一个小包。
    \item AC (Air Canada):加拿大最大的航司(?)但是国内线服务和廉航一样烂,大部分低价舱位没有里程积累,也没有托运和登机行李。美加之间的航线还可以,但是本人睡似过去什么也没吃到……
    \item AS (Alaska Alrlines):虽然总部在阿拉斯加,但在旧金山周围有不少航线,飞机上提供免费小点心和咖啡,体验尚可
    \item F9 (Frontier):极致的货物体验……如果你想省钱,又不赶时间,提前入手一张F9的票挺好的,但有概率动不动就延误,入手联程票务必提前看看航班准点率。理所当然的,行李和机上服务全部没有,甚至纸质登机牌也要钱
    \item UA (United):美国老牌航司,票价性价比和服务都中规中矩,有些国内坐不到的神秘机型,飞友可以打卡
    \item NK (Spirit): 比F9稍微好一些,至少可以免费获取纸质登机牌,但行李和服务也是不存在的,票价便宜
    \item WN (Southwest):【2024年末】很有特色的航司,不指定座位,只给出登机顺序,座位先到先选。国内线免费赠送两个托运行李。【2025年】免费托运政策有调整,详见\href{https://support.southwest.com/helpcenter/s/article/policy-changes},据说这个独有的登机方式也有可能修改
    \item AA (American):和UA,DL等等一样,干线航班是本公司自营,支线有可能是外包的,只坐过LAX-SFO的支线小飞机,体验一般,也没有托运和机上服务。
    \item B6 (jetblue):机上有不限速免费wifi,票价不是特别低的那一档,但体验确实不错
    \item DL (Delta):同上,有wifi,此外Delta有一款印着logo的小饼干,蛮好吃的。喜欢收集的还可以问机组要免费的trading cards,印有各种机型
    \item HA (Hawaiian):美国大陆\~夏威夷航线上会提供很有夏威夷特色的餐品,本人买的是SFO-HNL-HND的国际联程,所以前半段有行李额度,单买美国国内段的行李额度不甚清楚
    
\end{itemize}

\subsection{铁路旅行}
美国铁路客运并不发达,线路较为固定,主要以Amtrak营运的全国性铁路网络和一些零散的地方铁路组成。旧金山附近的ACE和CalTrain都不属于Amtrak系统,但坐51B直达的Berkeley火车站是Amtrak所辖的。以下主要介绍Amtrak,末尾会略微介绍CalTrain。

Amtrak实行和机票一样的浮动票价,早买会便宜很多,同时有学生专属的promo code(可Google搜索),在APP购票时输入可获得一定程度的优惠。购票后直接凭电子车票二维码上车即可。如果你想收集Amtrak的纸质车票,你需要去一个够大的车站打印——比如Berkeley附近的Emeryville站。大部分小车站和国内的乘降所一样,无安检无售票处无检票口,上车后由列车员检票。美国大部分列车都不指定座位号,座位的使用遵循先来后到原则,如果要在热门假期抢一个靠窗座位还是有些难度的。

【待补充】


\section{美国境内}
持有效J-1签证在美国境内旅游是合法的。美国的旅游资源非常丰富:在旧金山、洛杉矶、拉斯维加斯、纽约、波士顿等体验都市风情,在一号公路、阿拉斯加、丹佛、美西大环线等自驾线路领略独特的自然风光。\\ \indent 春季学期的春假,秋季学期的国庆节、感恩节,平时课业较少的时候都可以合理安排旅游。湾区绝大部分地方都可以乘BART前往;加州境内可以选择租车自驾,推荐budget、dollar、avis等租车公司,避免车况很差的小公司。前往洛杉矶、圣迭戈可以乘Amtrak海滨火车。\\ \indent 较远可以乘飞机。短期背包旅游建议Spirit、Frontier等无免费行李额的廉航,便宜,各项配置都很基础。如果带了行李箱但没买行李额会有高额罚款。Southwest属于廉航但是整体配置较好,还有两个免费行李额。长途飞行不建议廉航,座椅舒适度很差。机票和酒店可以在航司官网和国内外代理平台(携程、去哪儿、飞猪、Expedia、Booking、Klook等)比价。奥克兰机场OAK离学校最近,旧金山机场SFO和圣何塞机场SJC较远,建议多人拼uber。另外,使用谷歌地图搜索地名时建议使用英文,UI和语音导航设置为中文。


\subsection{湾区及周边}
    \begin{itemize}
        \item 谷歌地图与攻略 \\
        $\ast$ \href{https://maps.app.goo.gl/4eEPoS2CreDBuatw7}{湾区及周边坐标点收藏}
        \item 旧金山 \\
        著名景点与体验项目有金门大桥、联合广场、铛铛车、唐人街、渔人码头、Alcatraz Island等。联合广场附近有梅西百货等众多好店;唐人街会在中秋、10.01中国国庆时组织集市、游行、文艺表演等各种活动,平时也可以前往逛街或吃中餐。Alcatraz Island是旧金山北部的一个孤岛,岛上监狱曾用于关押重刑犯,现已被改造为观光景点。旧金山近年治安恶化,建议结伴同行。
        \item 奥克兰 \\
        治安很差,特色景点少,不建议独自前往。驾照路考一般安排在奥克兰车管所,教练和考官关系很好,不犯原则性错误很容易通过。\\
        \quad $\ast$ \href{https://luyang-yu.notion.site/Oakland-Coliseum-f32d2dcbb91f44a5bd5a20ff099e454a?pvs=4}{笔者整理的路考经验}
        \item Emeryville \\
        Emeryville是奥克兰南部的一个小城市。其中AMC影院在湾区排片较多并且价格相对实惠。影院附近的Marina公园的日落相当不错。影院周围还有大量潮牌店,适合逛街。不过该区域治安堪比奥克兰,笔者曾在AMC影院亲历枪击现场。
        \item 硅谷 \\
        谷歌、苹果、微软等世界一流大厂和斯坦福大学云集于此。可通过官网或小红书查看是否开放参观、是否需要预约。周边人口不算密集,适合散步放空。
        \item 一号公路 \\
        加州自驾首选路线,海滨公路驾车赏景非常惬意。沿途可经停半月湾和Poplar Beach、鸽点灯塔、蒙特雷等。一号公路部分路段会经常关闭,出发前请查询是否封路。去程可通过添加途经点设置路线为一号公路,返程可走推荐的最快路线。如只有一天时间推荐开到蒙特雷返回,如小长假出行可考虑开至洛杉矶或圣迭戈,也可加钱异地还车。
        \item 纳帕 \\
        坐落于湾区北部,作为世界著名的葡萄酒产区,纳帕开设了大量红酒庄园和葡萄地,非常适合一天自驾。绝大部分酒庄必须提前在官网预定参观和品酒,并且大同小异,一天安排两个就行。笔者推荐Beringer Vineyards,价格实惠且酒的品质很好。最有名的“作品一号”Opus One价格较高且难预约,不过可以去门口拍照。另外Castello de Amorosa和Domaine Carneros也很出片。回程可以顺路拍金门大桥。 \\
        $\ast$   根据联邦法,美国本土最低购酒、饮酒年龄为21周岁。 \\
        $\ast$   根据加州车辆法,加州酒驾标准非常宽松(中国醉驾未达加州酒驾标准),但还是建议品酒后休息几小时再开车。
        \item 国家公园 \\
        湾区附近著名的有优胜美地国家公园和美洲杉(红杉)国家公园。一般需要结伴自驾。
    \end{itemize}

\subsection{洛杉矶}
    \begin{itemize}
        \item 谷歌地图与攻略 \\
        $\ast$ \href{https://maps.app.goo.gl/9ZG6N5wpqbBRAzxu8}{洛杉矶坐标点收藏} \\
        洛杉矶面积很大,各个景点之间较远且公共交通不便,建议预约机场租车。
        \item 好莱坞星光大道 \\
        美国影视圈和演艺圈诸多名人的地砖映射着独特的美式风情。周边还有奥斯卡金像奖的颁奖地柯达剧院、中国戏院等地标。
        \item 比弗利山庄 \\
        洛杉矶富人区,比弗利花园适合散步。
        \item UCLA \\
        位于比佛利山庄附近。与伯克利分校相比面积更大,建筑极具特色。
        \item 格里菲斯天文台与好莱坞标志 \\
        天文台里有介绍天体的生成和演化过程,有模拟各个星球体重秤、望远镜等设施。在天文台可以与远处的好莱坞标志合影。天文台车位非常紧张,基本都是停在盘山公路旁。
        \item 华纳兄弟娱乐公司 \\
        可以现场购买观光票,导游带队乘坐游览车参观电影工厂。一边听取导游讲解,一边亲身步入许多经典电影的取景地,同时还会参观电影制作和道具工坊、华纳兄弟水塔等地标,感受美国成熟的电影工业。附近有洛杉矶环球影城,票很难抢。
        \item 洛杉矶市区 \\
        小东京、韩国城满街的日韩招牌让人瞬时穿越到东京和首尔。非常推荐对应的日韩料理,性价比高。Urban Light是一大片灯柱阵列,晚上非常出片。6th St Bridge铭刻着洛杉矶的光辉和黑暗历史,具有独特的拱门结构,曾作为《终结者》、《变形金刚》等的取景地。Downtown附近治安较差。
        \item Harbor Breeze Cruises \\
        乘船出海看鲸和海豚,需要提前在官网订票。船上工作人员会在接近鲸和海豚聚集区时提醒乘客,但有可能因为天气等原因没法看到。靠近港口的地方停泊着玛丽皇后号邮轮。
        \item 圣莫尼卡 \\
        有非常著名的海滩和66号公路终点路牌。黄昏时可以看日落和各种街头表演。\\
         \item 洛杉矶游记(流水账) \\
         路不拾遗伯克利,夜不闭户洛圣都。作为加州最大的城市和美国人口第二大城市,洛杉矶像是将国内的城市布局进行横向延展,纵向压缩。在飞机还未落地前,我们已经可以从舷窗中眺望到这一如棋盘分布众星云集的大都市风貌。
好莱坞星光大道是众多游客前往打卡驻足之地,以其铺设着印有美国各行业巨星名字的星形砖块而闻名。这条长约1.3英里的大道,以其独特的方式向在电影、电视、音乐、广播和娱乐行业中做出杰出贡献的人们致敬,也为到访的旅行者展现着美国娱乐业的蓬勃发展。位于洛杉矶西北方向的格里菲斯天文台,更是提供了一个远眺好莱坞标志的平台。\\
提到娱乐业,洛杉矶无疑是美国的一颗明珠。而华纳兄弟也是其中相当璀璨的一员。成立于19世纪初成立的华纳兄弟,几经易主与收购,在一个世纪的磨砺与突破中,为全球无数观众奉献了许多高质量的电影和动画作品。在这里,你可以用几十美金乘上华纳兄弟工坊的游览车,在道具工厂中窥见电影角色的奇幻装备在银幕后的真实形态;在取景小镇中走进电影角色的家中,感受被经典情节的场景全方位环抱的亲临感;在声音工厂中聆听拟音师的讲解,揭秘如何用我们日常生活中随手可及的物件打造出电影里引人入胜的神奇音效。\\
相比好莱坞、华纳兄弟而言,在制作洛杉矶的旅行攻略时,你很少能看到有关洛杉矶夜景的推荐。洛杉矶的夜大多充斥着川流不息的车道,嘈杂的胎噪和眩目的车灯,但你也能在这嘈杂喧嚣之中寻得一方宝地。作为洛杉矶夜晚的打卡圣地,洛杉矶郡艺术博物馆门口的Urban Light吸引力无数旅行者。大家在方正交错的灯柱间欣赏光影倾泻而下的斑驳,踱步其中,灯柱阵列的几何变换不断点燃着美学思考的火花。作为La La Land的取景地之一,Urban Light也谱写着塞巴斯蒂安和米娅在这座城市相遇,在唱唱跳跳中坠入爱河,在追梦路上扶持慰藉的爱情与人生赞歌。\\
在Downtown LA,第六街桥横跨洛杉矶河。这座于2022年7月才启用的大桥,成为了洛杉矶的一处新地标。作为这一大都会的脉络,它连接着城市的心脏和远方的梦境。当夜幕降临,桥上灯火璀璨,闪烁着城市的活力。迎面而来的微风轻抚,带着洛杉矶12月的清凉,沉浸在这个城市的韵律之中。桥上车流涌动,行人匆匆,每一辆车都是一个故事的开端,每一位行人都是城市的诗篇。但是,由于大桥附近的社区治安问题频发,这座宏伟地标也曾一度被勒令短期封闭。有关第六街桥的沿革和未来也随着洛杉矶的夜幕逐渐幻化迷离。\\
作为伯克利的访学学生,怎么能不悄悄潜入UCLA,体验加州大学的另一种可能?正午时分的UCLA在阳光辉映下充满活力。在Royce Hall的长廊中,光束与阴影被廊柱分隔开,长廊一侧的浮雕光影斑驳,给一味的纵深感平添了几分趣味。而长廊对面的主图书馆,则讲述着中世纪欧洲的古典美学。\\
洛杉矶旁的城市long beach,是美国西海岸说唱的摇篮之一,同时也是出海看鲸鱼和海豚的绝佳之地。这次我们乘船从长滩的aquarium港口出发,在日落时分探寻太平洋东岸的海洋赞歌。\\
夜幕降临,我们选择用久违的东亚风情结束洛杉矶之行。漫步于坐落在Downtown的小东京,感觉自己仿佛穿越到远东一隅。小桥流水,樱花飘落,每一帧都如同一幅和风画卷。而在小东京附近的韩国城,传统的韩屋和现代的时尚店铺鳞次栉比。美食街汇聚了各种韩国特色小吃店,让人不禁流连。\\
这座充满活力和创意的城市,每一处都蕴含着不同的风情。在这片充满梦想的土地上,肆意沉浸在城市的多彩景观和文化的交融之中,仿佛找到了属于自己的一部分,成为这座城市故事中的丰富插图一笔。
    \end{itemize}

    \subsection{拉斯维加斯}
    \begin{itemize}
        \item 谷歌地图与攻略 \\
        $\ast$ \href{https://maps.app.goo.gl/iEPgmg8PCUejdZho8}{拉斯维加斯坐标点收藏} \\
        主要地标都在The Strip赌城大道附近,建议定周边酒店,方便公交出行。不过公交速度很慢,也可以选择租车。同时,拉斯维加斯也常作为美西大环线的起止点。
    \end{itemize}

    \subsection{美东:奥兰多、纽约、波士顿}
    \begin{itemize}
        \item 谷歌地图与攻略 \\
        $\ast$ \href{https://maps.app.goo.gl/MZNaQabDQErWrCzN7}{美东坐标点收藏} \\
    \end{itemize}

    \subsection{阿拉斯加}
    \begin{itemize}
        \item 谷歌地图与攻略 \\
        $\ast$ \href{https://maps.app.goo.gl/vSeUm3Fnf78kuK1H9}{阿拉斯加坐标点收藏} \\
        $\ast$ \href{https://docs.qq.com/sheet/DSFVSYnNuWmh4Y3VT}{阿拉斯加自制攻略} \\
    \end{itemize}

    \subsection{丹佛}
    \begin{itemize}
        \item 谷歌地图与攻略 \\
        $\ast$ \href{https://maps.app.goo.gl/icLPvppHWGdQvnYH6}{丹佛坐标点收藏} \\
        $\ast$ \href{https://docs.qq.com/sheet/DSGJiU2FsYVpnR1RH}{丹佛自制攻略} \\
    \end{itemize}

    \subsection{美西大环线}
    \begin{itemize}
        \item 谷歌地图与攻略 \\
        $\ast$ \href{https://maps.app.goo.gl/CcrQn4tGMGdN5zgY7}{美西大环线坐标点收藏} \\
        $\ast$ \href{https://docs.qq.com/sheet/DSFBqdnpUUWxFQ2tR}{美西大环线自制攻略} \\
        \item 美西大环线游记(流水账) \\
美西大环线作为美国最有名的自驾线路之一,一直是众多旅行者的向往之地。
大环线的第一站便是位于犹他州西部的锡安国家公园。虽然它常常在美国国家公园榜单中跻身前三,但是网上对它的评价却相对两极分化。这次我们自驾从Zion Mt-Carmel Highway穿行经过锡安国家公园。\\
然后,经过两小时车程便到达附近的布莱斯峡谷国家公园。由于海拔较高,在深冬季节,布莱斯常常被冰雪覆盖。但是我们这次并没有发现很厚的冰层。在没有冰爪的情况下徒步从sunset point,经过Navajo Loop和Queen's garden到达sunrise point。徒步近两小时的过程中,红石、峡谷和稀疏的森林交错出现。另外,inspiration和Bryce两个观景台可以将布莱斯峡谷一览无余。正午时分,你可以感受到阳光倾泻到峡谷的壮阔。
从布莱斯自驾穿过圆顶礁国家公园,可以到达最为著名的拱门国家公园。圆顶礁的路旁被少量冰雪覆盖,二十分钟车程后,路旁的植物逐渐披上金黄外衣,再经过半小时车程,景色逐渐变得翠绿,在深秋谱写一曲春意物语。又经过半小时,道路两旁基本没有植被,仅有一些稀疏的矮灌木,和更远处耸立的灰色礁石,仿佛置身于另外一个陌生星球。\\
拱门国家公园最为称道就是精致拱门,从Wolfe Ranch到Delicate Arch的徒步过程相当具有挑战性。Trail绵延数公里,路面起伏非常大,甚至这里没有被严格隔离出一条规整的步道,所以经常需要在岩石块,碎沙地和悬崖边寻找落脚点。但是最后置身于巨型拱门之间,感觉之前的努力并没有白费,颇有轻舟已过万重山的快意淋漓。虽说“上山容易下山难”,但是在返回途中偶遇一家麋鹿,也是重复景色中的一些点缀。\\
与拱门国家公园相邻的峡谷地国家公园,以一个更加广阔的视角俯瞰莫阿布小镇和科罗拉多河。在grand view point,我们发现了一处形似巨兽踩下的巨坑,它是由多年河流冲刷和风霜雨雪侵蚀形成的谷地。它的形状实在过于生动,以至于我们都希望为它赋予一个古老的传说。在返程途中,我们还发现了mesa拱门,著名摄影师乔瓦尼·西蒙尼在此拍摄的一张作品,被选作为Windows7系统的主题壁纸之一。我们到达的时候已经日落,mesa拱门在夕阳熹微的夜幕中呈现出和壁纸不一样的色彩,像是蒙上了一层尘埃,诉说着游人罕至的寂寥。\\
从莫阿布小镇向西南方向进入亚利桑那州,纪念碑谷一定是旅行者不能错过的胜地。这里曾是电影阿甘正传末尾阿甘跑步穿行美国情节的取景地。进入内部,一条蜿蜒且危险的砂石路是自驾游览纪念碑谷的唯一路径。纪念碑谷不同于之前的几个国家公园,它更多地保留了峡谷本来的样貌,并且作为纳瓦霍美洲原住民保护区,这里的人文也展现出不一样的风情。\\
继续向东可以抵达美西大环线上的一个必经小镇——佩奇。它主要是因为毗邻马蹄湾、羚羊谷、鲍威尔湖而闻名。马蹄湾是科罗拉多河在佩奇小镇的一个270度大转弯处,由于形似马蹄而得名。在这里,有你一定不能错过的日落。鲍威尔湖作为美国第二大人工湖,在湖水侵蚀作用下形成了比美国西部太平洋海岸线还长的湖岸线。在鲍威尔湖,你既可以体验到瓦威普的宁静闲适,也能感受到美国第二大坝格兰峡谷大坝的壮阔。\\
向南深入亚利桑那州,大峡谷国家公园的壮丽是旅行者们不能错过的。沿着bright angel trail可以逐渐迈入谷底。据说,居住在此的霍皮族原住民曾骑马沿着这条trail去谷底的科罗拉多河获取淡水,但是由于它紧靠悬崖,坡度很大,有许多原住民命丧于此。公园官方也在trail的入口处也对游客进行了警告。\\
大峡谷国家公园为美西大环线的国家公园部分画上了一个句号,继续向西可以进入美国66号公路。这条被誉为“母亲之路”的公路曾在大萧条和二战时期为美国的旅游业和军工做出了极大贡献,也是无数移民与拓荒者穿越探索北美大陆的通道。但1956年时任总统艾森豪威尔推出《洲际公路法案》,新建成更加快速便捷的州际公路取代了66号公路的地位。我们在沿途的塞利格曼小镇也感受到随着66号公路的衰落带来的荒凉。作为电影《赛车总动员》的原型小镇,如今只剩下几面涂鸦墙和几个周边店在诉说着66号公路往日的辉煌。\\
美西大环线作为人们探索美国西部风貌的桥梁,展现了一副高耸山川与广阔平原交融,现代文明与历史传统杂糅的壮美画卷。拉斯维加斯的纸醉金迷与美西的风光旖旎看似水火不容,但他们却能在引擎的一声声咆哮与嘶吼中逐渐靠近、相交,为旅行者展现出视觉和思想的双重冲击。

\end{itemize}

\section{其他国家或地区}
J-1签证扩充了不少免签国家,但有限的假期和繁杂的signature手续让访学期间多国旅行相对困难。不过可以利用访学结束至返校的一段时间,藉由航班中转前往其他国家或地区。往东飞可选择土耳其、巴尔干各国、新加坡等,往西飞可选择日本、韩国、中国台湾、中国香港等。截至2024年8月,新加坡、塞尔维亚、阿尔巴尼亚、波黑对华免签;黑山跟团可免签;土耳其网上申请立即下签;日韩签证利用本科生身份办理难度较低;自海外过境香港返回内地可免签注逗留7日。这些行程的选择可以结合个人兴趣和机票价格,同时需要确认只中转不出境是否需要签证(或落地签)。

\subsection{加拿大}
在美国访学期间,利用学签和伯克利的学生身份,很容易申请通过加拿大10年旅行签。相比国内较长的申请周期和很高的默拒概率,不失为一个不错的办理加签机会。全过程分为三步,大约需要3-6周,参考时间线:
\begin{description}
    \item[2023.2.2] 网上提交材料,收到submission confirmation
    \item[2023.2.3] 收到指纹信(BIL)
    \item[2023.2.6] 在SF的ASC录指纹
    \item[2023.2.13-2.14] 收到第二封指纹信$\to$重新录指纹(因笔者无指纹,通常不会发生)
    \item[2023.2.17] 指纹状态显示已完成
    \item[2023.2.24] 收到要求寄护照信(OPR)
    \item[2023.2.25-2.27] USPS寄出护照USPS显示送达
    \item[2023.3.1] 收到VFS确认邮件$\to$显示护照转交IRCC
    \item[2023.3.3] 收到IRCC确认邮件,签证状态approved
    \item[2023.3.6-3.7-3.9] VFS邮件通知护照寄出$\to$ USPS上可查到包裹$\to$护照寄达  
\end{description}
参考攻略:北美票帝\url{https://piao.tips/canada-visa-in-the-us/amp/}
    
\paragraph*{IRCC网申} 在网申前,请先检查护照的有效期。加拿大签证的有效期与护照有效期一致,一本全新的十年有效护照可以最大化加拿大十年旅游签的有效时间。同时,护照是在美最重要的身份证件,\textbf{本流程中会有1周以上的时间护照不在身边},请确保期间不会用到护照作为身份证明,并时刻关注护照的动向。网申在IRCC官网完成,可以参考北美票帝的攻略。注意,旅行计划中需要提供往返机票,可以订两张能够全额退票的机票,随后退掉。

\paragraph*{线下录指纹} 收到指纹信(BIL)后,可选择VAC或ASC线下录指纹。VAC相对空一些,但是只在纽约和洛杉矶有。ASC可以在\url{https://my.uscis.gov/appointmentscheduler-appointment/ca/en}查看地址,并进行预约。网站上往往只能预约1个月以后的时间。在\url{https://piao.tips/canada-asc-walk-in}中,有实时更新(最新的在最下面)每个ASC录指纹的情况,可选择离自己近的、支持walk in的ASC。在网站上预约这个ASC较晚的录指纹申请,随后\textbf{带预约信提前前往},如笔者预约了3月初SF的ASC,在2.6早上ASC刚开门时walk in,也顺利完成了录指纹。需带材料:
\begin{itemize}
    \item 护照原件
    \item 录指纹信(BIL)
    \item 预约单
\end{itemize}

\paragraph*{寄护照贴签} 收到OPR信后,需把护照寄到洛杉矶或纽约贴签,加州选择洛杉矶较快。在官网上提供了FedEx的邮寄服务,图方便可以选择,考虑性价比的情况下\textbf{不推荐}。USPS的邮寄费用最便宜并且处理速度很快,和伯克利邮局的工作人员说明情况,他们会耐心提供指导。需准备的材料(此处仅列出,具体信息详见北美票帝攻略):
\begin{itemize}
    \item 需在邮局购买\begin{itemize}
        \item money order
        \item 寄去的信封(写好地址,寄出时付款)
        \item 寄回的信封(买信封时一并买好邮票,信封上填好地址,装入寄去的信封)
    \end{itemize}
    \item 需自行准备\begin{itemize}
        \item 护照原件
        \item OPR信打印件
        \item TT consent form(2份,签名)
        \item contact info(A4打印,包括姓名,地址,电话和邮箱)
        \item 护照生物信息页复印件(2份)
    \end{itemize}
\end{itemize}

\subsection{土耳其与巴尔干半岛}
$\ast$ \href{https://docs.qq.com/sheet/DSGVQamFlUG9IdWlx}{土耳其与巴尔干自制攻略} \\